\chapter{Configuration Files}
\label{ch:configuration_files}
The framework is configured with human-readable key-value based configuration files.
The configuration format consists of section headers within $[$ and $]$ brackets, and a global section without a header at the beginning.
Each of these sections contains a set of key-value pairs separated by the \texttt{=} character.
Comments are indicated using the hash symbol (\texttt{\#}).

Since configuration files are highly user-specific and do not directly belong to the \corry framework, they should not be stored in the \corry repository.
However, working examples can be found in the \dir{testing/} directory of the repository.

\corry can handle any file extensions for geometry and configuration files.
The examples, however, follow the convention of using the extension \texttt{*.conf} for both the detector and configuration files.

The framework has two required layers of configuration files:
\begin{itemize}
  \item The \textbf{main} configuration: It is passed directly to the binary and contains both the global framework configuration and the list of modules to instantiate, together with their configuration.
  \item The \textbf{detector} configuration: Passed to the framework to determine the geometry.
  Describes the detector setup and contains the position, orientation and type of all detectors along with additional properties crucial for the reconstruction.
\end{itemize}

In the following paragraphs, the available types and the unit system are explained and an introduction to the different configuration files is given.

\section{Parsing types and units}
\label{sec:config_values}
The \corry framework supports the use of a variety of types for all configuration values.
The module requesting the configuration key specifies how the value type should be interpreted.
An error will be raised if either a necessary key is not specified in the configuration file, the conversion to the desired type is not possible, or if the given value is outside the domain of possible options.
Please refer to the module documentation in Chapter~\ref{ch:modules} for the list of module parameters and their types.
The value is parsed in an intuitive manner, however a few special rules do apply:
\begin{itemize}
\item If the value is a \textbf{string}, it may be enclosed by a single pair of double quotation marks (\texttt{"}), which are stripped before passing the value to the module(s).
If the string is not enclosed by quotation marks, all whitespace before and after the value is erased.
If the value is an array of strings, the value is split at every whitespace or comma (\texttt{,}) that is not enclosed in quotation marks.
\item If the value is a \textbf{boolean}, either numerical (\texttt{0}, \texttt{1}) or textual (\texttt{false}, \texttt{true}) representations are accepted.
\item If the value is a \textbf{relative path}, that path will be made absolute by adding the absolute path of the directory that contains the configuration file where the key is defined.
\item If the value is an \textbf{arithmetic} type, it may have a suffix indicating the unit.
The list of base units is shown in Table~\ref{tab:units}.
\end{itemize}

The internal base units of the framework are not chosen for user convenience, but for maximum precision of the calculations and to avoid the necessity of conversions in the code.
Combinations of base units can be specified by using the multiplication sign \texttt{*} and the division sign \texttt{/} that are parsed in linear order (thus $\frac{V m}{s^2}$ should be specified as $V*m/s/s$).
The framework assumes the default units (as given in Table~\ref{tab:units}) if the unit is not explicitly specified.

\begin{warning}
  If no units are specified, values will always be interpreted in the base units of the framework.
  In some cases this can lead to unexpected results.
  E.g. specifying a pixel pitch as \parameter{pixel_pitch = 55,55} results in a detector with a pixel size of \SI{55 x 55}{\milli \meter}.
  Therefore, it is strongly recommended to always specify the units explicitly for all parameters that are not dimensionless in the configuration files.
\end{warning}

\begin{table}[tbp]
\caption{List of units supported by \corry}
\label{tab:units}
\centering
\begin{tabular}{lll}
  \toprule
\textbf{Quantity}                 & \textbf{Default unit}                   & \textbf{Auxiliary units} \\
 \midrule
\multirow{6}{*}{\textit{Length}}  & \multirow{6}{*}{mm (millimeter)}        & nm (nanometer)           \\
                                  &                                         & um (micrometer)          \\
                                  &                                         & cm (centimeter)          \\
                                  &                                         & dm (decimeter)           \\
                                  &                                         & m (meter)                \\
                                  &                                         & km (kilometer)           \\
 \midrule
\multirow{4}{*}{\textit{Time}}    & \multirow{4}{*}{ns (nanosecond)}        & ps (picosecond)          \\
                                  &                                         & us (microsecond)         \\
                                  &                                         & ms (millisecond)         \\
                                  &                                         & s (second)               \\
\midrule
\multirow{3}{*}{\textit{Energy}}  & \multirow{3}{*}{MeV (megaelectronvolt)} & eV (electronvolt)        \\
                                  &                                         & keV (kiloelectronvolt)   \\
                                  &                                         & GeV (gigaelectronvolt)   \\
\midrule
\textit{Temperature}              & K (kelvin)                              & ---                      \\
\midrule
\multirow{3}{*}{\textit{Charge}}  & \multirow{3}{*}{e (elementary charge)}  & ke (kiloelectrons)       \\
                                  &                                         & fC (femtocoulomb)        \\
                                  &                                         & C (coulomb)              \\
\midrule
\multirow{2}{*}{\textit{Voltage}} & \multirow{2}{*}{MV (megavolt)}          & V (volt)                 \\
                                  &                                         & kV (kilovolt)            \\
\midrule
\textit{Magnetic field strength}  & T (tesla)                               & mT (millitesla)                 \\
\midrule
\multirow{2}{*}{\textit{Angle}}   & \multirow{2}{*}{rad (radian)}           & deg (degree)             \\
                                  &                                         & mrad (milliradian)       \\
\bottomrule
\end{tabular}
\end{table}

Examples of specifying key-values pairs of various types are given below:
\begin{minted}[frame=single,framesep=3pt,breaklines=true,tabsize=2,linenos]{ini}
# All whitespace at the front and back is removed
first_string =   string_without_quotation
# All whitespace within the quotation marks is preserved
second_string = "  string with quotation marks  "
# Keys are split on whitespace and commas
string_array = "first element" "second element","third element"
# Integers and floats can be specified in standard formats
int_value = 42
float_value = 123.456e9
# Units can be passed to arithmetic type
energy_value = 1.23MeV
time_value = 42ns
# Units are combined in linear order
acceleration_value = 1.0m/s/s
# Thus the two quantities below have the same units
random_quantity_a = 1.0deg*kV/m/s*K
random_quantity_b = 1.0deg*kV*K/m/s
# Relative paths are expanded to absolute
# Path below will be /home/user/test/ if the config file is in /home/user
output_path = "test/"
# Booleans can be represented in numerical or textual style
my_switch = true
my_other_switch = 0
\end{minted}

\subsection{File format}
\label{sec:config_file_format}
Throughout the framework, a simplified version of TOML~\cite{tomlgit} is used as standard format for configuration files.
The format is defined as follows:
\begin{enumerate}
\item All whitespace at the beginning or end of a line are stripped by the parser.
In the rest of this format specification, \textit{line} refers to the line with this whitespace stripped.
\item Empty lines are ignored.
\item Every non-empty line should start with either \texttt{\#}, \texttt{[} or an alphanumeric character.
Every other character should lead to an immediate parse error.
\item If the line starts with a hash character (\texttt{\#}), it is interpreted as comment and all other content on that line is ignored.
\item If the line starts with an open square bracket (\texttt{[}), it indicates a section header (also known as configuration header).
The line should contain a string with alphanumeric characters and underscores indicating the header name, followed by a closing square bracket (\texttt{]}) to end the header.
After any number of ignored whitespace characters there could be a \texttt{\#} character.
If this is the case, the rest of the line is handled as specified in point~3.
Otherwise, there should not be any other character on the line that is not whitespace.
Any line that does not comply to these specifications should lead to an immediate parse error.
Multiple section headers with the same name are allowed.
All key-value pairs in the line following this section header are part of this section until a new section header is started.
\item If the line starts with an alphanumeric character, the line should indicate a key-value pair.
The beginning of the line should contain a string of alphabetic characters, numbers, dots (\texttt{.}), colons (\texttt{:}), and/or underscores (\texttt{\_}), but it may only start with an alphanumeric character.
This string indicates the 'key'.
After an optional number of ignored whitespace, the key should be followed by an equality sign (\texttt{$=$}).
Any text between the \texttt{$=$} and the first \texttt{\#} character not enclosed within a pair of single or double quotes (\texttt{'} or \texttt{"}) is known as the non-stripped string.
Any character after the \texttt{\#} is handled as specified in point 3.
If the line does not contain any non-enclosed \texttt{\#} character, the value ends at the end of the line instead.
The 'value' of the key-value pair is the non-stripped string with all whitespace in front and at the end stripped.
The value may not be empty.
Any line that does not comply to these specifications should lead to an immediate parse error.
\item The value may consist of multiple nested dimensions that are grouped by pairs of square brackets (\texttt{[} and \texttt{]}).
The number of square brackets should be properly balanced, otherwise an error is raised.
Square brackets that should not be used for grouping should be enclosed in quotation marks.
Every dimension is split at every whitespace sequence and comma character (\texttt{,}) not enclosed in quotation marks.
Implicit square brackets are added to the beginning and end of the value, if these are not explicitly added.
A few situations require the explicit addition of outer brackets such as matrices with only one column element, i.e. with dimension 1xN.
\item The sections of the value that are interpreted as separate entities are named elements.
For a single value the element is on the zeroth dimension, for arrays on the first dimension, and for matrices on the second dimension.
Elements can be forced by using quotation marks, either single or double quotes (\texttt{'} or \texttt{"}).
The number of both types of quotation marks should be properly balanced, otherwise an error is raised.
The conversion to the elements to the actual type is performed when accessing the value.
\item All key-value pairs defined before the first section header are part of a zero-length empty section header.
\end{enumerate}

\subsection{Accessing parameters}
\label{sec:accessing_parameters}
Values are accessed via the configuration object.
In the following example, the key is a string called \parameter{key}, the object is named \parameter{config} and the type \parameter{TYPE} is a valid \CPP type that the value should represent.
The values can be accessed via the following methods:
\begin{minted}[frame=single,framesep=3pt,breaklines=true,tabsize=2,linenos]{c++}
// Returns true if the key exists and false otherwise
config.has("key")
// Returns the number of keys found from the provided initializer list:
config.count({"key1", "key2", "key3"});
// Returns the value in the given type, throws an exception if not existing or a conversion to TYPE is not possible
config.get<TYPE>("key")
// Returns the value in the given type or the provided default value if it does not exist
config.get<TYPE>("key", default_value)
// Returns an array of elements of the given type
config.getArray<TYPE>("key")
// Returns a matrix: an array of arrays of elements of the given type
config.getMatrix<TYPE>("key")
// Returns an absolute (canonical if it should exist) path to a file, where the second input value determines if the existence of the path is checked
config.getPath("key", true /* check if path exists */)
// Return an array of absolute paths, where the second input value determines if the existence of the paths is checked
config.getPathArray("key", false /* do not check if paths exists */)
// Returns the value as literal text including possible quotation marks
config.getText("key")
// Set the value of key to the default value if the key is not defined
config.setDefault("key", default_value)
// Set the value of the key to the default array if key is not defined
config.setDefaultArray<TYPE>("key", vector_of_default_values)
// Create an alias named new_key for the already existing old_key or throws an exception if the old_key does not exist
config.setAlias("new_key", "old_key")
\end{minted}

Conversions to the requested type are using the \parameter{from_string} and \parameter{to_string} methods provided by the framework string utility library.
These conversions largely follow standard \CPP parsing, with one important exception.
If (and only if) the value is retrieved as a C/\CPP string and the string is fully enclosed by a pair of \texttt{"} characters, these are stripped before returning the value.
Strings can thus also be provided with or without quotation marks.

\begin{warning}
    It should be noted that a conversion from string to the requested type is a comparatively heavy operation.
    For performance-critical sections of the code, one should consider fetching the configuration value once and caching it in a local variable.
\end{warning}

\section{Main configuration}
\label{sec:main_config}
The main configuration file consists of a set of sections specifying the modules to be used.
All modules are executed in the \emph{linear} order in which they are defined.
There are a few section names that have a special meaning in the main configuration, namely the following:
\begin{itemize}
\item The \textbf{global} (framework) header sections: These are all zero-length section headers (including the one at the beginning of the file) and all sections marked with the header \texttt{[Corryvreckan]} (case-insensitive).
These are combined and accessed together as the global configuration, which contains all parameters of the framework itself as described in Section~\ref{sec:framework_parameters}.
All key-value pairs defined in this section are also inherited by all individual configurations as long the key is not defined in the module configuration itself. This is encouraged for module parameters used in multiple modules.
\item The \textbf{ignore} header sections: All sections with name \texttt{[Ignore]} are ignored.
Key-value pairs defined in the section, as well as the section itself, are discarded by the parser.
These section headers are useful for quickly enabling and disabling individual modules by replacing their actual name by an ignore section header.
\end{itemize}

All other section headers are used to instantiate modules of the respective name.
Installed module libraries are loaded automatically at startup.
Parameters defined under the header of a module are local to that module and are not inherited by other modules.

An example for a valid albeit illustrative \corry main configuration file is:
\begin{minted}[frame=single,framesep=3pt,breaklines=true,tabsize=2,linenos]{ini}
# Key is part of the empty section and therefore the global configuration
+random_string = "example1"
# The location of the detector configuration is a global parameter
detectors_file = "testbeam_setup.conf"
# The Corryvreckan section is also considered global and merged with the above
[Corryvreckan]
another_random_string = "example2"

# Stop after one thousand events:
number_of_events = 1000

# First section runs "ModuleA"
[ModuleA]
# This module takes no parameters

# Ignore this second section:
[Ignore]
my_key = "my_value"

# Third section runs "ModuleC" with configured parameters:
[ModuleC]
int_value = 2
vector_of_doubles = 23.0, 45.6, 78.9
\end{minted}

\section{Detector configuration}
\label{sec:detector_config}
The detector configuration file consists of a set of sections that describe the detectors in the setup.
Each section starts with a header describing the name used to identify the detector; all names are required to be unique.
Every detector should contain all of the following parameters:
\begin{itemize}
\item The \parameter{role} parameter is an array of strings indicating the function(s) of the respective detector. This can be \parameter{dut}, \parameter{reference} (\parameter{ref}), \parameter{auxiliary} (\parameter{aux}), or \parameter{none}, where the latter is the default. With the default role, the respective detector participates in tracking but is neither used as reference plane for alignment and correlations, nor treated as DUT. In a reference role, the detector is used as anchor for relative alignments and its position and orientation is used for comparison when producing correlation and residual plots. As DUT, the detector is by default excluded from tracking, and all DUT-type modules are executed for this detector. As an auxiliary device, the detector may provide additional information but does not partake in the reconstruction. This is useful to e.g. include trigger logic units (TLUs) providing only timing information.
\begin{warning}
There always has to be exactly \emph{one} reference detector in the setup. For setups with a single detector only, the role should be configured as \parameter{dut, reference} for the detector to act as both. Auxiliary devices cannot have any other role simultaneously.
\end{warning}

\item The \parameter{type} parameter is a string describing the type of detector, e.g.\ \parameter{Timepix3} or \parameter{CLICpix2}. This value might be used by some modules to distinguish between different types.
\item The \parameter{position} in the world frame.
This is the position of the geometric center of the sensor given in world coordinates as X, Y and Z as defined in Section~\ref{sec:coordinate_systems}.
\item An \parameter{orientation_mode} that determines the way that the orientation is applied.
This can be either \texttt{xyz}, \texttt{zyx}, or \texttt{zxz}, where \textbf{\texttt{xyz}} is used as default if the parameter is not specified. Three angles are expected as input, which should always be provided in the order in which they are applied.
\begin{itemize}
  \item The \texttt{xyz} option uses extrinsic Euler angles to apply a rotation around the global $X$ axis, followed by a rotation around the global $Y$ axis, and finally a rotation around the global $Z$ axis.
  \item The \texttt{zyx} option uses the \textbf{extrinsic Z-Y-X} convention for Euler angles, also known as Pitch-Roll-Yaw or 321 convention. The rotation is represented by three angles describing an initial rotation of an angle $\phi$ (yaw) about the $Z$ axis, followed by a rotation of an angle $\theta$ (pitch) about the initial $Y$ axis, followed by a third rotation of an angle $\psi$ (roll) about the initial $X$ axis.
  \item The \texttt{zxz} option uses the \textbf{extrinsic Z-X-Z} convention for Euler angles. This option is also known as the 3-1-3 or the "x-convention" and the most widely used definition of Euler angles~\cite{eulerangles}.
\end{itemize}
\begin{warning}
It is highly recommended to always explicitly state the orientation mode, rather than relying on the default configuration, to avoid unwanted behavior.
\end{warning}

\item The \parameter{orientation} to specify the Euler angles in logical order (e.g. first $X$, then $Y$, then $Z$ for the \texttt{xyz} method), interpreted using the method above. An example for three Euler angles would be:
\begin{minted}[frame=single,framesep=3pt,breaklines=true,tabsize=2,linenos]{ini}
orientation_mode = "zyx"
orientation = 45deg 10deg 12deg
\end{minted}
which describes a rotation of \SI{45}{\degree} around the $Z$ axis, followed by a \SI{10}{\degree} rotation around the initial $Y$ axis, and finally a rotation of \SI{12}{\degree} around the initial $X$ axis.
\begin{warning}
All supported rotations are extrinsic active rotations, i.e. the vector itself is rotated, not the coordinate system. All angles in configuration files should be specified in the order they will be applied.
\end{warning}

\item The \parameter{number_of_pixels} parameter represents a two-dimensional vector with the number of pixels in the active matrix in the column and row directions respectively.
\item The \parameter{pixel_pitch} is a two-dimensional vector defining the size of a single pixel in the column and row directions respectively.
\item The intrinsic resolution of the detector has to be specified using the \parameter{spatial_resolution} parameter, a two-dimensional vector holding the position resolution for the column and row directions. This value is used to assign the uncertainty of cluster positions. This parameter defaults to the pitch$/\sqrt{12}$ of the respective detector if not specified.
\item The intrinsic time resolution of the detector should be specified using the \parameter{time_resolution} parameter with units of time. This can be used to apply detector specific time cuts in modules. This parameter is only required when using relative time cuts in the analysis.
\item The \parameter{time_offset} can be used to shift the reference time frame of an individual detector to e.g.\ account for time of flight effects between different detector planes by adding a fixed offset.
\item Pixels to be masked in the offline analysis can be placed in a separate file specified by the \parameter{mask_file} parameter, which is explained in detail in Section~\ref{sec:masking}.
\item A region of interest in the given detector can be defined using the \parameter{roi} parameter. More details on this functionality can be found in Section~\ref{sec:roi}.
\end{itemize}

An example configuration file describing a setup with one CLICpix2~\cite{clicpix2,clicpix2-pisa} detector named \parameter{016_CP_PS} and two Timepix3~\cite{timepix3} detectors (\parameter{W0013_D04} and \parameter{W0013_J05}) is the following:

\begin{minted}[frame=single,framesep=3pt,breaklines=true,tabsize=2,linenos]{ini}
[W0013_D04]
number_of_pixels = 256, 256
orientation = 9deg, 9deg, 0deg
orientation_mode = "xyz"
pixel_pitch = 55um, 55um
position = 0um, 0um, 10mm
spatial_resolution = 4um,4um
time_resolution = 3ns
type = "Timepix3"

[016_CP_PS]
mask_file = "mask_016_CP_PS.conf"
number_of_pixels = 128,128
orientation = -0.02deg, 0.0deg, -0.015deg
orientation_mode = "xyz"
pixel_pitch = 25um, 25um
position = -0.9mm, 0.21mm, 106.0mm
spatial_resolution = 8um,8um
time_resolution = 1ms
role = "dut"
type = "CLICpix2"

[W0013_J05]
number_of_pixels = 256, 256
orientation = -9deg, 9deg, 0deg
orientation_mode = "xyz"
pixel_pitch = 55um, 55um
position = 0um, 0um, 204mm
spatial_resolution = 4um,4um
time_resolution = 3ns
role = "reference"
type = "Timepix3"
\end{minted}

\subsection{Masking Pixels Offline}
\label{sec:masking}

Mask files can be provided for individual detectors, which allows the user to mask specific pixels in the reconstruction.
+The following syntax is used for each line within the mask file:
\begin{itemize}
    \item \command{c COL}: masking all pixels in column \parameter{COL}
    \item \command{r ROW}: masking all pixels in row \parameter{ROW}
    \item \command{p COL ROW}: masking the single pixel at address \parameter{COL, ROW}
\end{itemize}

\begin{warning}
It should be noted that the individual event loader modules have to take care of discarding masked pixels manually, the \corry framework only parses the mask file and attaches the mask information to the respective detector. The event loader modules should thus always query the detector object for masks before adding new pixels to the data collections.
\end{warning}

\subsection{Defining a Region of Interest}
\label{sec:roi}

The region of interest (ROI) feature of each detector allows tracks or clusters to be marked as within a certain region on the respective detector.
This information can be used in analyses to restrict the selection of tracks or clusters to certain regions of the device, e.g.\ to exclude known bad regions from the calculation of efficiencies.

The ROI is defined as a polynomial in local pixel coordinates of the device using the \parameter{roi} keyword. A rectangle could, for example, be defined by providing the four corners of the shape via the following:

\begin{minted}[frame=single,framesep=3pt,breaklines=true,tabsize=2,linenos]{ini}
roi = [1, 1], [1, 120], [60, 120], [60, 1]
\end{minted}

Internally, a winding number algorithm is used to determine whether a certain local position is within or outside the given polynomial shape.
Two functions are provided by the detector API:

\begin{minted}[frame=single,framesep=3pt,breaklines=true,tabsize=2,linenos]{c++}
// Returns "true" if the track is found to be within the ROI
bool isWithinROI(const Track* track);
// Returns "true" if the cluster, as well as all its constituent pixels, are found to be within the ROI
bool isWithinROI(const Cluster* cluster);
\end{minted}
