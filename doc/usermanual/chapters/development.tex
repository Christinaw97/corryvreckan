\chapter{Extending the \corry Framework}

This chapter provides some initial information for developers planning on extending the \corry framework.
\corry is a community project that benefits from active participation in the development and code contributions from users.
Users are encouraged to discuss their needs via the issue tracker of the repository~\cite{corry-issue-tracker} to receive ideas and guidance on how to implement a specific feature.
Getting in touch with other developers early in the development cycle avoids spending time on features which already exist or are currently under development by other users.

The repository contains a few tools to facilitate contributions and to ensure code quality as detailed in Chapter~\ref{ch:testing}.

\section{Coding and Naming Conventions}

The code base of the \corry is well-documented and follows concise rules on naming schemes and coding conventions.
This enables maintaining a high quality of code and ensures maintainability over a longer period of time.
In the following, some of the most important conventions are described.
In case of doubt, existing code should be used to infer the coding style from.

\subsection{Naming Schemes}

The following coding and naming conventions should be adhered to when writing code which eventually should be merged into the main repository.

\begin{description}
    \item[Namespace] The \parameter{corryvreckan} namespace is to be used for all classes which are part of the framework, nested namespaces may be defined. It is encouraged to make use of \command{using namespace corryvreckan;} in implementation files only for this namespace. Especially the namespace \parameter{std} should always be referred to directly at the function to be called, e.g.\ \command{std::string test}. In a few other cases, such as \parameter{ROOT::Math}, the \command{using} directive may be used to improve readability of the code.

    \item[Class names] Class names are typeset in CamelCase, starting with a capital letter, e.g.\ \command{class ModuleManager{}}. Every class should provide sensible Doxygen documentation for the class itself as well as for all member functions.

    \item[Member functions] Naming conventions are different for public and private class members. Public member function names are typeset as camelCase names without underscores, e.g.\ \command{getTimeResolution()}. Private member functions use lower-case names, separating individual words by an underscore, e.g.\ \command{create_detector_modules(...)}. This allows to visually distinguish between public and restricted access when reading code.

    In general, public member function names should follow the \command{get}/\command{set} convention, i.e.\ functions which retrieve information and alter the state of an object should be marked accordingly. Getter functions should be made \parameter{const} where possible to allow usage of constant objects of the respective class.

    \item[Member variables] Member variables of classes should always be private and accessed only via respective public member functions. This allows to change the class implementation and its internal members without requiring to rewrite code which accesses them. Member names should be typeset in lower-case letters, a trailing underscore is used to mark them as member variables, e.g.\ \parameter{bool terminate_}. This immediately sets them apart from local variables declared within a function.
\end{description}

\subsection{Formatting}

A set of formatting rules is applied to the code base in order to avoid unnecessary changes from different editors and to maintain readable code.
It is vital to follow these rules during development in order to avoid additional changes to the code, just to adhere to the formatting.
There are several options to integrate this into the development workflow:

\begin{itemize}
  \item Many popular editors feature direct integration either with \command{clang-format} or their own formatting facilities.
  \item A build target called \command{make format} is provided if the \command{clang-format} tool is installed. Running this command before committing code will ensure correct formatting.
  \item This can be further simplified by installing the \emph{git hook} provided in the directory \dir{/etc/git-hooks/}. A hook is a script called by \command{git} before a certain action. In this case, it is a pre-commit hook which automatically runs \command{clang-format} in the background and offers to update the formatting of the code to be committed. It can be installed by calling
  \begin{minted}[frame=single,framesep=3pt,breaklines=true,tabsize=2,linenos]{bash}
  ./etc/git-hooks/install-hooks.sh
  \end{minted}
  once.
\end{itemize}
The formatting rules are defined in the \file{.clang-format} file in the repository in machine-readable form (for \command{clang-format}, that is) but can be summarized as follows:

\begin{itemize}
  \item The column width should be 125 characters, with a line break afterwards.
  \item New scopes are indented by four whitespaces, no tab characters are to be used.
  \item Namespaces are indented just as other code is.
  \item No spaces should be introduced before parentheses ().
  \item Included header files should be sorted alphabetically.
  \item The pointer asterisk should be left-aligned, i.e. \command{int* foo} instead of \command{int *foo}.
\end{itemize}
The continuous integration automatically checks if the code adheres to the defined format as described in Section~\ref{sec:ci}.

\section{Writing Additional Modules}

Given the modular structure of the framework, its functionality can be easily extended by adding a new module.
To facilitate the creation of new modules including their CMake files and initial documentation, the script \file{addModule.sh} is provided in the \dir{etc/} directory of the repository.
It will ask for a name and type of the module as described in Section~\ref{sec:module_manager} and create all code necessary to compile a first (and empty) version of the files.

The content of each of the files is described in detail in the following paragraphs.

\subsection{Files of a Module}
\label{sec:module_files}
Every module directory should at minimum contain the following documents (with \texttt{<ModuleName>} replaced by the name of the module):
\begin{itemize}
\item \textbf{\file{CMakeLists.txt}}: The build script to load the dependencies and define the source files of the library.
\item \textbf{\file{README.md}}: Full documentation of the module.
\item \textbf{\file{<ModuleName>.h}}: The header file of the module.
\item \textbf{\file{<ModuleName>.cpp}}: The implementation file of the module.
\end{itemize}
These files are discussed in more detail below.
By default, all modules added to the \dir{src/modules/} directory will be built automatically by CMake.
If a module depends on additional packages which not every user may have installed, one can consider adding the following line to the top of the module's \file{CMakeLists.txt}:
\begin{minted}[frame=single,framesep=3pt,breaklines=true,tabsize=2,linenos]{cmake}
CORRYVRECKAN_ENABLE_DEFAULT(OFF)
\end{minted}

\paragraph{CMakeLists.txt}
Contains the build description of the module with the following components:
\begin{enumerate}
\item On the first line either \parameter{CORRYVRECKAN_DETECTOR_MODULE(MODULE_NAME)}, \parameter{CORRYVRECKAN_DUT_MODULE(MODULE_NAME)} or \parameter{CORRYVRECKAN_GLOBAL_MODULE(MODULE_NAME)} depending on the type of module defined.
The internal name of the module is automatically saved in the variable \parameter{${MODULE_NAME}} which should be used as an argument to other functions.
Another name can be used by overwriting the variable content, but in the examples below, \parameter{${MODULE_NAME}} is used exclusively and is the preferred method of implementation.
For DUT and Detector modules, the type of detector this module is capable of handling can be specified by adding so-called type restrictions, e.g.\
\begin{minted}[frame=single,framesep=3pt,breaklines=true,tabsize=2,linenos]{cmake}
CORRYVRECKAN_DETECTOR_TYPE(${MODULE_NAME} "Timepix3" "CLICpix2")
\end{minted}
The module will then only be instantiated for detectors of one of the given types. This is particularly useful for event loader modules which read a very specific file format.

\item The following lines should contain the logic to load possible dependencies of the module (below is an example to load Geant4).
Only ROOT is automatically included and linked to the module.
\item A line with \texttt{\textbf{CORRYVRECKAN\_MODULE\_SOURCES(\$\{MODULE\_NAME\} \textit{sources})}} defines the module source files. Here, \texttt{sources} should be replaced by a list of all source files relevant to this module.
\item Possible lines to include additional directories and to link libraries for dependencies loaded earlier.
\item A line containing \parameter{CORRYVRECKAN_MODULE_INSTALL(${MODULE_NAME})} to set up the required target for the module to be installed to.
\end{enumerate}

A simple \file{CMakeLists.txt} for a module named \parameter{Test} which should run only on DUT detectors of type \emph{Timepix3} is provided below as an example.
\vspace{5pt}

\begin{minted}[frame=single,framesep=3pt,breaklines=true,tabsize=2,linenos]{cmake}
# Define module and save name to MODULE_NAME
CORRYVRECKAN_DUT_MODULE(MODULE_NAME)
CORRYVRECKAN_DETECTOR_TYPE(${MODULE_NAME} "Timepix3")

# Add the sources for this module
CORRYVRECKAN_MODULE_SOURCES(${MODULE_NAME}
    Test.cpp
)

# Provide standard install target
CORRYVRECKAN_MODULE_INSTALL(${MODULE_NAME})
\end{minted}

\paragraph{README.md}
The \file{README.md} serves as the documentation for the module and should be written in Markdown format~\cite{markdown}.
It is automatically converted to \LaTeX~using Pandoc~\cite{pandoc} and included in the user manual in Chapter~\ref{ch:modules}.
By documenting the module functionality in Markdown, the information is also viewable with a web browser in the repository within the module sub-folder.

The \file{README.md} should follow the structure indicated in the \file{README.md} file of the \parameter{Dummy} module in \dir{src/modules/Dummy}, and should contain at least the following sections:
\begin{itemize}
\item The H1-size header with the name of the module and at least the following required elements: the \textbf{Maintainer}, the \textbf{Module Type} and the \textbf{Status} of the module.
The module type should be either \textbf{GLOBAL}, \textbf{DETECTOR}, \textbf{DUT}.
If the module is working and well-tested, the status of the module should be \textbf{Functional}.
By default, new modules are given the status \textbf{Immature}.
The maintainer should mention the full name of the module maintainer, with their email address in parentheses.
A minimal header is therefore:
\begin{verbatim}
# ModuleName
Maintainer: Example Author (<example@example.org>)
Module Type: GLOBAL
Status: Functional
\end{verbatim}
In addition, the \textbf{Detector Type} should be mentioned for modules of types \textbf{DETECTOR} and \textbf{DUT}.
\item An H3-size section named \textbf{Description}, containing a short description of the module.
\item An H3-size section named \textbf{Parameters}, with all available configuration parameters of the module.
The parameters should be briefly explained in an itemised list with the name of the parameter set as an inline code block.
\item An H3-size section named \textbf{Plots Created}, listing all plots created by this module.
\item An H3-size section with the title \textbf{Usage} which should contain at least one simple example of a valid configuration for the module.
\end{itemize}

\paragraph{ModuleName.h and ModuleName.cpp}
All modules should consist of both a header file and a source file.
In the header file, the module is defined together with all of its methods.
Doxygen documentation should be added to explain what each method does.
The source file should provide the implementation of every method.
Methods should only be declared in the header and defined in the source file in order to keep the interface clean.

\subsection{Module structure}
\label{sec:module_structure}
All modules must inherit from the \parameter{Module} base class, which can be found in \dir{src/core/module/Module.hpp}.
The module base class provides two base constructors, a few convenient methods and several methods which the user is required to override.
Each module should provide a constructor using the fixed set of arguments defined by the framework; this particular constructor is always called during by the module instantiation logic.
These arguments for the constructor differ for global and detector/DUT modules.

For global modules, the constructor for a \module{TestModule} should be:
\begin{minted}[frame=single,framesep=3pt,breaklines=true,tabsize=2,linenos]{c++}
TestModule(Configuration& config, std::vector<std::shared_ptr<Detector>> detectors): Module(std::move(config), detectors) {}
\end{minted}

For detector and DUT modules, the first argument are the same, but the last argument is a \texttt{std::shared\_ptr} to the linked detector.
It should always forward this detector to the base class together with the configuration object.
Thus, the constructor of a detector module is:
\begin{minted}[frame=single,framesep=3pt,breaklines=true,tabsize=2,linenos]{c++}
TestModule(Configuration& config, std::shared_ptr<Detector> detector): Module(std::move(config), std::move(detector)) {}
\end{minted}

In addition to the constructor, each module can override the following methods:
\begin{itemize}
\item \parameter{initialise()}: Called after loading and constructing all modules and before starting the analysis loop.
This method can for example be used to initialize histograms.
\item \parameter{run(std::shared_ptr<Clipboard> clipboard)}: Called for every time frame or triggered event to be analyzed. The argument represents a pointer to the clipboard where the event data is stored.
A status code is returned to signal the framework whether to continue processing data or to end the run.
\item \parameter{finalise()}: Called after processing all events in the run and before destructing the module.
Typically used to summarize statistics like the number of tracks used in the analysis or analysis results like the chip efficiency.
Any exceptions should be thrown from here instead of the destructor.
\end{itemize}
